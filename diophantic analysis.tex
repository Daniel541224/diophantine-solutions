\documentclass[11pt,a4paper]{article}

\usepackage[spanish,es-noquoting,es-tabla]{babel}
\usepackage[T1]{fontenc}
\usepackage[utf8]{inputenc}

\usepackage{newtxtext,newtxmath}
\usepackage[protrusion,expansion]{microtype}

\usepackage[a4paper,margin=1in]{geometry}
\setlength{\parindent}{1.2em}
\setlength{\parskip}{0.2em}
\linespread{1.05}

\usepackage{mathtools,amssymb,amsthm,amsfonts}
\usepackage{enumitem}
\setlist[itemize]{topsep=3pt,itemsep=2pt,parsep=0pt}
\setlist[enumerate]{topsep=3pt,itemsep=2pt,parsep=0pt}

\usepackage{hyperref}
\hypersetup{
  colorlinks=true,
  linkcolor=black,
  citecolor=black,
  urlcolor=blue!40!black,
  pdfauthor={Daniel Carrillo},
}

\usepackage{fancyhdr}
\pagestyle{fancy}
\fancyhf{}
\renewcommand{\headrulewidth}{0pt}
\fancyfoot[C]{\thepage}


\theoremstyle{plain}
\newtheorem{theorem}{Theorem}[section]
\newtheorem{lemma}[theorem]{Lemma}
\newtheorem{proposition}[theorem]{Proposition}
\newtheorem{corollary}[theorem]{Corollary}

\theoremstyle{definition}
\newtheorem{definition}[theorem]{Definition}
\newtheorem{example}[theorem]{Example}

\theoremstyle{remark}
\newtheorem{remark}[theorem]{Remark}


\newcommand{\scaps}[1]{\textsc{#1}}
\newcommand{\TopRule}{\rule{\linewidth}{0.9pt}}
\newcommand{\MidRule}{\rule{0.78\linewidth}{0.8pt}}

\title{}
\author{}
\date{}

\begin{document}
\begin{titlepage}
  \centering
  \vspace*{1.4cm}
  {\Large Universidad Industrial de Santander\par}
  \vspace{0.3cm}
  {\large \scaps{School of Mathematics}\par}
  \vspace{1.4cm}
  \TopRule\par
  \vspace{0.9cm}
  {\bfseries\LARGE Constructive method for linear Diophantine\\[-0.1em]
  \LARGE equations in three variables\par}
  \vspace{0.4cm}
  {\large (Based on Bézout's lemma)\par}
  \vspace{0.9cm}
  \MidRule\par
  \vspace{1.8cm}

  \vspace{2.2cm}
  \vfill
\end{titlepage}


\setcounter{page}{1}


\begin{center}
  {\Large\bfseries Abstract}
\end{center}

\vspace{0.4em}
\noindent
This work presents an explicit and constructive procedure to describe all integer solutions
of the linear Diophantine equation $a x + b y + c z = d$. The approach relies on Bézout's lemma and
on the interpretation of $a x + b y + c z$ as a homomorphism $\varphi_n : \mathbb{Z}^3 \to \mathbb{Z}$ with kernel of rank $2$, so
each solution set is a coset $x_0 + L$. The main result is stated as a theorem and is translated into
a two-step algorithm: fix one variable and solve the two-variable equation by means of Bézout, obtaining
the complete parametrization in terms of two integers. The geometric reading (parallel integral planes)
and the primitive case $\gcd(a,b,c)=1$, where the recipe simplifies, are included. As an outlook, the
extension of the scheme to $n$ variables is sketched for future development.

\vspace{1.0cm}
\hrule
\vspace{1.0cm}


\begin{center}
  {\Large\bfseries Framework}
\end{center}

\vspace{0.4em}
\noindent
The starting point of this work is the relationship between the greatest common divisor (gcd) and the least
common multiple (lcm), both supported by the algebraic structure of \emph{integer linear combinations}.
Bézout's lemma states that for every pair of integers $a,b$ there exists a combination
$a x_0 + b y_0 = \gcd(a,b)$, which shows that the set of all common multiples and divisors of
$a$ and $b$ is organized by means of a simple linear equation in $\mathbb{Z}^2$. In this way, the notion of divisibility
is translated into the existence of integer solutions of a linear identity, establishing the algebraic
foundation of every linear Diophantine equation.

When passing to three variables, the interest no longer lies only in checking the existence of solutions,
but in \emph{describing them completely}. The problem becomes one of \emph{parametrization}: how to express
all possible solutions by means of a minimal number of integer parameters, so that each particular solution
can be obtained by substituting suitable values. In this context, “to parametrize well” means to preserve
the underlying algebraic structure—the subgroup of homogeneous solutions—and at the same time to keep
the method constructive.

This work does not introduce new theory in the strict sense; it relies entirely on already established
tools, but organizes them in such a way that they produce an \emph{explicit and direct} description of the general
solution of the linear Diophantine equation in three variables. The proposed approach combines
Bézout's lemma, the notion of the kernel of a homomorphism, and the free structure of subgroups of $\mathbb{Z}^n$.
In this way one obtains an operative method that allows us to pass from abstract theory to an effective construction
of all solutions.

From a geometric point of view, the equation
\[
a x + b y + c z = d
\]
defines a family of parallel integral planes in $\mathbb{Z}^3$. Each plane corresponds to a coset
$x_0 + L$, where $L$ is the homogeneous subgroup of solutions of $a x + b y + c z = 0$. This geometric
reading motivates the interpretation of the main result as a complete characterization of that family of
planes and of the integral lattice that generates them.

\begin{remark}[Relationship between Bézout and Diophantine equations]
Bézout's lemma may be interpreted as the fundamental case of a linear Diophantine equation in two variables.
Every linear equation
\(a_1x_1 + \cdots + a_n x_n = d\)
can be solved recursively by reducing it to an equation in two variables of the form
\(ax + by = d'\), which is precisely Bézout's form
(an elementary exposition of this case may be found in \cite{Niven}).
Thus, the general method does not introduce new ideas, but rather amplifies the original principle of the lemma.

\end{remark}


\vspace{1.2cm}
\hrule
\vspace{1.2cm}


\begin{center}
  {\Large\bfseries Origin and personal motivation}
\end{center}

\vspace{0.4em}
\noindent
This work arises from a sustained interest in Bézout's lemma and in the structural role it plays in arithmetic.
Although it is usually presented as an elementary result, its reach is much greater: it connects the arithmetic
notion of greatest common divisor with an algebraic structure of linear combinations and, at the same time,
with a geometric reading in terms of integral lines and planes.

Since my studies on the associativity of the least common multiple, I have worked systematically with
Bézout's lemma, finding in it a unifying principle between the arithmetic, the algebraic, and the geometric.
This fascination is what drives the present work: to show that, starting from this fundamental identity,
one can construct explicit and beautifully simple methods to describe general solutions of Diophantine
equations, without resorting to new theory, but by reorganizing the essential elements that Bézout himself
leaves us.



\vspace{1.5cm}
\hrule
\vspace{1.5cm}

\begin{center}
  {\Large\bfseries Contextualization and comparison of methods}
\end{center}

\vspace{0.4em}
\noindent
During the development of this topic in class, a method proposed by another student was presented to deal
with linear Diophantine equations in three variables, and the rest were invited to reflect on its validity.
At that moment I presented the procedure that is developed here, based on Bézout's lemma, and was told that
it was not correct. However, the subsequent analysis revealed that the method put forward in this work is
not only valid, but actually constitutes a general formulation, whereas the one proposed on that occasion
corresponded to a particular case.

The technique of \emph{grouping terms} into a single intermediate variable—for example, substituting
$w=4x+7y$ in the equation $8x+14y+5z=11$ to reduce it to $2w+5z=11$—appears occasionally in different
sources and classical number-theory handbooks. Nevertheless, in those texts this reduction is usually
presented as an operative device without further theoretical development. In general, elementary treatments
of linear Diophantine equations tend to focus exclusively on the case of two variables, while the versions
for three or more variables are mentioned only marginally, merely stating that the existence criterion
associated with the greatest common divisor extends to $n$ variables, but without offering a complete proof.

The present work seeks to fill this conceptual gap: to establish an explicit description of the equation in
three variables as a natural step within the same Bézout framework, showing that the jump from two to three
variables does not require new theory, but rather a coherent reorganization of already known arithmetic ideas.

\vspace{1.5cm}
\hrule
\vspace{1.5cm}


\section{Main result}

\noindent\textbf{Brief context.}
Based on Bézout's lemma and on the structure of the subgroup of solutions in $\mathbb{Z}^3$,
we obtain an explicit (constructive) description of all integer solutions of $ax+by+cz=d$.

\begin{theorem}[Integer solutions of $ax+by+cz=d$]
Let $a,b,c,d\in\mathbb{Z}$, with $(a,b)\neq (0,0)$, and let $g=\gcd(a,b)$.
If $\gcd(a,b,c)\mid d$, then the equation
\[
ax+by+cz=d
\]
has integer solutions, and all of them are given by
\[
\begin{cases}
x = x_0\,\dfrac{d-ct}{g} + \dfrac{b}{g}\,k,\\[0.6em]
y = y_0\,\dfrac{d-ct}{g} - \dfrac{a}{g}\,k,\\[0.6em]
z = t,
\end{cases}
\qquad k\in\mathbb{Z},\;\; t\in t_0 + \dfrac{g}{h}\mathbb{Z},
\]
where $(x_0,y_0)$ is a Bézout solution of $ax+by=g$.
Moreover, the set of solutions is a coset $x_0 + L$, where $L$ is a free subgroup of $\mathbb{Z}^3$ of rank $2$.


\end{theorem}

\noindent\textbf{Note.} If $(a,b)=(0,0)$, the equation reduces to $cz=d$. It has an integer solution if and only if
$c\mid d$, in which case $z=d/c$ and $x,y$ are free in $\mathbb{Z}$. This is the degenerate case and is treated
separately.


\begin{proof}
Let $h=\gcd(a,b,c)$. If $h\nmid d$, there is no solution. Suppose $h\mid d$ and fix
$g=\gcd(a,b)$. Choose $(x_0,y_0)\in\mathbb Z^2$ such that $a x_0+b y_0=g$ (Bézout).

\medskip
\noindent\textit{(I) Reminder: two variables.}
For $D\in\mathbb Z$, the equation $a x+b y=D$ has an integer solution $\iff g\mid D$.
If $g\mid D$, write $D=gq$; then $(x_1,y_1)=(q x_0, q y_0)$ is a solution.
If $(x,y)$ and $(x',y')$ are solutions, then $a(x-x')+b(y-y')=0$, whence
$(x-x',y-y')=\big(\tfrac{b}{g}k,\,-\tfrac{a}{g}k\big)$ for some $k\in\mathbb Z$
(because $\gcd(a/g,b/g)=1$). Thus,
\[
\text{all solutions are }(x,y)=\Big(x_0\tfrac{D}{g}+\tfrac{b}{g}k,\;\;
y_0\tfrac{D}{g}-\tfrac{a}{g}k\Big),\quad k\in\mathbb Z.
\tag{$\ast$}
\]

\medskip
\noindent\textit{(II) Three variables by reduction to two.}
Given $t\in\mathbb Z$, solving $a x+b y+c z=d$ with $z=t$ is equivalent to
\[
a x+b y=d-ct.
\]
By (\(\ast\)), this is possible $\iff g\mid(d-ct)$. Since $\gcd(c,g)=\gcd(a,b,c)=h$ and $h\mid d$,
there exists $t_0\in\mathbb Z$ satisfying $c t_0\equiv d\pmod{g}$, and \emph{all} admissible $t$ are
\[
t=t_0+\frac{g}{h}\,s,\qquad s\in\mathbb Z.
\]
For such $t$, the quotient $\frac{d-ct}{g}$ is an integer and, by (\(\ast\)), all solutions are given by
\[
x=x_0\,\frac{d-ct}{g}+\frac{b}{g}\,k,\qquad
y=y_0\,\frac{d-ct}{g}-\frac{a}{g}\,k,\qquad
z=t,\qquad k\in\mathbb Z.
\]

\medskip
In other words,
\[
\boxed{\;
x=x_0\,\frac{d-ct}{g}+\frac{b}{g}\,k,\quad
y=y_0\,\frac{d-ct}{g}-\frac{a}{g}\,k,\quad
z=t,\quad k\in\mathbb Z,\;\; t\in t_0+\tfrac{g}{h}\mathbb Z.
\;}
\]
\medskip

\noindent\textit{(III) Structure.}
The set of solutions is a coset of $\ker\varphi$, where
\[
\varphi:\mathbb{Z}^3 \to \mathbb{Z}, \qquad \varphi(x,y,z)=ax+by+cz.
\]
Since $\ker\varphi$ is a free subgroup of $\mathbb{Z}^3$ of rank $2$, there exists a particular solution
$(x_0,y_0,t_0)$ such that
\[
\{(x,y,z)\in\mathbb{Z}^3 : ax+by+cz=d\} = (x_0,y_0,t_0) + \ker\varphi.
\]

\medskip
In other words, the set of solutions is a coset $x_0 + L$, where
$L=\ker\varphi$ is a free subgroup of $\mathbb{Z}^3$ of rank $2$.
\end{proof}




\begin{corollary}[Constructive form in two steps]
Let $g=\gcd(a,b)$. Then all integer solutions of $ax+by+cz=d$ are obtained as follows:
\begin{enumerate}
    \item Fix $z=t\in\mathbb{Z}$ such that $g \mid (d-ct)$ and solve $ax+by=d-ct$;
    \item express the solutions as
    \[
        x = x_1 + \frac{b}{g}k,\qquad
        y = y_1 - \frac{a}{g}k,\qquad k\in\mathbb{Z},
    \]
    where $(x_1,y_1)$ is a particular solution of $ax+by=d-ct$, and recover $z=t$.
\end{enumerate}

\end{corollary}


\begin{example}
For $5x+4y+10z=8$, $g=\gcd(5,4)=1$, $(x_0,y_0)=(1,-1)$.
Then
\[
(x,y,z)=(8-10t+4k,\; -8+10t-5k,\; t),\quad k,t\in\mathbb{Z}.
\]
\end{example}

\begin{remark}[Geometry and limiting cases]
Let $g=\gcd(a,b)$ and $h=\gcd(a,b,c)$. The set of solutions is a coset
of a free subgroup of $\mathbb{Z}^3$ of rank $2$, generated by the direction vectors
\[
v_1=\Big(\tfrac{b}{g},-\tfrac{a}{g},0\Big)
\quad\text{and some}\quad
v_2=(u,v,\tfrac{g}{h})\in\ker\varphi,
\]
where $au+bv=-c\,\tfrac{g}{h}$. In particular, there exists a $v_2$ with third component $1$
\textbf{if and only if} $g\mid c$.

More precisely, if $(x_0,y_0,t_0)$ is a particular solution of $ax+by+cz=d$, then
the complete set of solutions is
\[
(x,y,z)=(x_0,y_0,t_0)+\langle v_1, v_2\rangle,
\]
which shows that geometrically the solution space is an integral plane in $\mathbb{Z}^3$.

If $c=0$, one recovers the two-variable case $ax+by=d$; if $g=1$, the family simplifies.
\end{remark}


\begin{remark}[Parallel family and image of $\varphi_n$]
Let $n = (a,b,c)$ and $h = \gcd(a,b,c)$. Consider the homomorphism
\[
\varphi_n : \mathbb{Z}^3 \to \mathbb{Z}, \qquad \varphi_n(x,y,z) = n \cdot (x,y,z) = a x + b y + c z.
\]
This map has image $h\mathbb{Z}$ and kernel $L = \ker \varphi_n$ of rank $2$.
For each $d \in h\mathbb{Z}$, the set of solutions of $n \cdot x = d$ is a \textit{coset}
$x_0 + L$, that is, an integral plane “filled” by the lattice $L$.

If $\gcd(a,b,c) = 1$, then $\mathrm{im}(\varphi_n) = \mathbb{Z}$ and all parallel
hyperplanes $n \cdot x = d$ that contain integer points appear (one for each $d$). If $h > 1$, only those parallels
with $d \in h\mathbb{Z}$ exist, spaced by steps of $h$ along the normal.

In particular, dividing the whole equation by $h$ yields the primitive form without changing the solution
set, whereas scaling only the coefficients by a factor $s\in\mathbb{Z}$ restricts
the possible values of $d$ to $s\,h\mathbb{Z}$.
\end{remark}


\clearpage

\begin{center}
  {\Large\bfseries Application and connection with exploratory problem 5.5 of \textit{Problemas de Teoría de Números} by professor Arnoldo Teherán}
\end{center}

\vspace{0.4em}

\begin{quote}
\small
\textbf{Problem.}
Similarly to the case of linear equations in linear algebra, one can consider systems of linear Diophantine equations.
For simplicity, the following system in three variables is proposed:
\[
\begin{cases}
a_{11}x_1 + a_{12}x_2 + a_{13}x_3 = b_1,\\[0.4em]
a_{21}x_1 + a_{22}x_2 + a_{23}x_3 = b_2,
\end{cases}
\qquad a_{ij},b_j\in\mathbb{Z},
\]
and one is asked to determine a criterion for this system to have a solution in $\mathbb{Z}$,
finding necessary and sufficient conditions to be able to eliminate one variable and obtain a
Diophantine equation in two variables that has a solution.
\end{quote}

\vspace{0.3em}
\noindent
This formulation is solved directly from the theoretical framework built in this work.
The \emph{elimination of a variable} corresponds precisely to fixing one variable in the general equation
\[
a x + b y + c z = d,
\]
reducing it to the two-variable case $a x + b y = d - c t$.
The proposed method shows that this reduction is possible whenever $\gcd(a,b,c)\mid d$,
and that the existence of solutions in $\mathbb{Z}$ is equivalent to the divisibility
\[
\gcd(a,b)\mid(d-c t),
\]
a condition that determines the admissible values of $t$.
Therefore, the “elimination criterion” requested in the problem is expressed in terms of Bézout's lemma:
the variable $z=t$ can be fixed freely in an arithmetic progression
$t\in t_0+\tfrac{g}{h}\mathbb{Z}$ (where $g=\gcd(a,b)$ and $h=\gcd(a,b,c)$),
and the remaining variables are determined constructively via Bézout's linear combinations.

\vspace{0.3em}
\noindent
Moreover, the same principle extends without conceptual modification to the case of
$m$ linear Diophantine equations in three variables,
and similarly can be formulated for a general system
of $m$ equations in $n$ variables.
Each equation
\[
a_{i1}x_1 + a_{i2}x_2 + \cdots + a_{in}x_n = b_i, \quad i=1,\dots,m,
\]
can be interpreted as a homomorphism
$\varphi_i:\mathbb{Z}^n\to\mathbb{Z}$,
and the complete system as the intersection of $m$ affine preimages
\[
\varphi_1^{-1}(b_1)\cap\cdots\cap\varphi_m^{-1}(b_m).
\]

In arithmetic terms, each elimination step must preserve the structure of $\mathbb{Z}^n$.
This is achieved by $\mathbb{Z}$-linear combinations (unimodular transformations)
that allow one to reduce the system without leaving the integer context; equivalently,
this procedure corresponds to applying the Smith normal form to the linear map associated with the system \cite{DummitFoote}.

\noindent
Consequently, the constructive method developed here describes not only the equation in three variables,
but also the general mechanism that allows one to address \emph{any finite linear Diophantine system},
repeating the reduction process until reaching an equation in two variables.
\vspace{1cm}

\begin{example}[Arithmetic elimination and structure of the system]
To illustrate this principle in action, consider the linear Diophantine system
\[
\begin{cases}
4x_1 + 6x_2 + 9x_3 = 7,\\[0.3em]
3x_1 - 5x_2 + 6x_3 = 4,
\end{cases}
\qquad x_1,x_2,x_3\in\mathbb{Z}.
\]

We apply a $\mathbb{Z}$-linear combination of the equations by means of the unimodular matrix
\[
U=\begin{pmatrix}2&-3\\[2pt]1&-1\end{pmatrix}\in \mathrm{GL}_2(\mathbb{Z}),\qquad \det(U)=1.
\]
Multiplying $U$ by the original system, we obtain
\[
\left\{
\begin{aligned}
&2(4x_1+6x_2+9x_3)-3(3x_1-5x_2+6x_3)=2,\\
&(4x_1+6x_2+9x_3)-(3x_1-5x_2+6x_3)=3,
\end{aligned}
\right.
\]
that is,
\[
\boxed{-x_1+27x_2=2}
\qquad\text{and}\qquad
\boxed{x_1+11x_2+3x_3=3}.
\]

The first equation allows us to eliminate $x_1$, giving $x_1 = 27x_2 - 2$. Substituting into the second,
\[
(27x_2 - 2) + 11x_2 + 3x_3 = 3
\;\Longrightarrow\;
38x_2 + 3x_3 = 5.
\]
Fixing $x_3 = t \in \mathbb{Z}$, the equation is reduced to the two-variable case:
\[
38x_2 = 5 - 3t.
\]
Therefore, there is a solution if and only if $\gcd(38,3)\mid(5 - 3t)$, which is equivalent to the congruence
\[
3t \equiv 5 \pmod{38}.
\]
Since $3^{-1}\equiv 13\pmod{38}$, we obtain $t \equiv 27 \pmod{38}$, that is,
\[
t = 27 + 38k,\qquad k \in \mathbb{Z}.
\]
For such values of $t$, we have $x_2 = \dfrac{5 - 3t}{38} = -2 - 3k$ and finally
\[
x_1 = 27x_2 - 2 = -56 - 81k.
\]

We conclude that the complete set of integer solutions of the system is
\[
(x_1,x_2,x_3) = (-56,-2,27) + k\,(-81,-3,38),
\qquad k \in \mathbb{Z}.
\]
This result confirms the theoretical structure established earlier: the system has a solution if and only if
the divisibility condition imposed by Bézout is satisfied, and the solution set forms a coset of a free
subgroup of $\mathbb{Z}^3$ of rank $1$, generated by a direction vector belonging to the kernel of the associated
linear map.

To verify this, it suffices to substitute the general solution into the original system:
\[
\begin{aligned}
4x_1 + 6x_2 + 9x_3
&= 4(-56 - 81k) + 6(-2 - 3k) + 9(27 + 38k) \\
&= -224 - 324k - 12 - 18k + 243 + 342k \\
&= 7, \\[0.6em]
3x_1 - 5x_2 + 6x_3
&= 3(-56 - 81k) - 5(-2 - 3k) + 6(27 + 38k) \\
&= -168 - 243k + 10 + 15k + 162 + 228k \\
&= 4.
\end{aligned}
\]
Therefore, the family of solutions obtained satisfies both equations for every $k\in\mathbb{Z}$.

\end{example}



\section{Primitive case \texorpdfstring{$\gcd(a,b,c)=1$}{gcd(a,b,c)=1}}

According to Remark~1.5, when the greatest common divisor of the coefficients is $1$,
the image of the homomorphism $\varphi_n(x)=a x + b y + c z$ is all of $\mathbb{Z}$.
This means that, for any $d \in \mathbb{Z}$,
the equation $a x + b y + c z = d$ has integer solutions.
In this case the equation is already in its \emph{primitive form}.
In particular, when moreover $\gcd(a,b)=1$ (so that there exists $(x_0,y_0)$ with $a x_0+b y_0=1$),
we have $g=\gcd(a,b)=1$ and the constraint $g\mid(d-ct)$ disappears: \emph{$t$ is free in $\mathbb{Z}$}.
The solution set retains the structure described in the previous remarks:
each plane $n \cdot x = d$ is a translate of the subgroup $L = \ker(\varphi_n)$,
and successive planes appear one for each integer $d$.

\begin{corollary}[Primitive case $\gcd(a,b,c)=1$ and $\gcd(a,b)=1$]
If $\gcd(a,b,c)=1$ and $\gcd(a,b)=1$, the equation
\[
a x + b y + c z = d
\]
has integer solutions for every $d \in \mathbb{Z}$,
and the general solution is obtained directly as
\[
\begin{cases}
x = x_0 (d - c t) + b k,\\[0.4em]
y = y_0 (d - c t) - a k,\\[0.4em]
z = t,
\end{cases}
\qquad k,t \in \mathbb{Z},
\]
where $(x_0, y_0)$ is a Bézout solution of $a x + b y = 1$; \textit{in particular, $t$ is free}.
\end{corollary}

\noindent\textit{Note.} If $\gcd(a,b,c)=1$ but $\gcd(a,b)=g>1$, the form of Theorem~1.1 remains valid and
$t$ runs through the progression $t\in t_0+\tfrac{g}{1}\mathbb{Z}=t_0+g\mathbb{Z}$,
since the condition $g\mid (d-ct)$ is not automatic in that case.

Geometrically, this case corresponds to an infinite family of integral planes
parallel to each other, each one “filled” by the rank-$2$ lattice $L$.
Each value of $d$ produces a distinct coset $x_0 + L$,
and all integers are attained as $d$ varies,
in contrast with the cases with $\gcd(a,b,c) > 1$, where only multiples of that value appear.

\begin{remark}
In practice, this is the most useful equation: elementary exercises and classic problems treated
in the academy are usually adapted precisely to this form.
It was from this formulation that the entire work was developed: first I obtained the equation
by a direct route, and subsequently I built the theoretical framework needed to justify
the neatness and algebraic coherence of its structure.
\end{remark}



\section*{Conclusion}

The expression obtained shows that the structure of the solutions of the equation
\[
a x + b y + c z = d
\]
arises naturally from Bézout's lemma, without the need for additional tools.
The procedure exhibits a recursive pattern: each new variable introduces
a congruence that preserves the form of the solutions and the free structure of the
homogeneous subgroup.

In this sense, the case of three variables is the first visible instance
of the general mechanism governing all linear Diophantine equations in
$\mathbb{Z}^n$: a chain of unimodular linear combinations
that leave the integer arithmetic intact.

This viewpoint suggests that Bézout's lemma is not just an isolated result,
but the generating principle of the entire linear Diophantine theory.



\begin{thebibliography}{9}

\bibitem{IrelandRosen}
K.~Ireland and M.~Rosen,
\textit{A Classical Introduction to Modern Number Theory},
2nd ed., Springer-Verlag, New York, 1990.

\bibitem{Niven}
I.~Niven, H.~S.~Zuckerman, H.~L.~Montgomery,
\textit{An Introduction to the Theory of Numbers},
5th ed., John Wiley \& Sons, 1991.

\bibitem{Apostol}
T.~M.~Apostol,
\textit{Introduction to Analytic Number Theory},
Springer, 1976.

\bibitem{DummitFoote}
D.~S.~Dummit and R.~M.~Foote,
\textit{Abstract Algebra},
3rd ed., John Wiley \& Sons, 2004.

\bibitem{Teheran}
A.~Teherán Herrera,
\textit{Problemas de Teoría de Números},
Universidad Industrial de Santander, 2025.
Available at:
\texttt{https://drive.google.com/...} (Accessed on 25-Oct-2025).

\bibitem{Zelator}
K.~Zelator,
\textit{Lattice Points on the Plane $ax+by+cz=d$ and the Diophantine System $ax+by+cz=d$, $ex+fy+gz=h$},
\texttt{arXiv:0805.1702} [math.GM], 2008.
Available at: \url{https://arxiv.org/abs/0805.1702}.

\bibitem{Kenaga}
B.~Kenaga,
\textit{Linear Diophantine Equations},
online Number Theory course, Millersville University.
Available at:
\url{https://sites.millersville.edu/bikenaga/number-theory/linear-diophantine-equations/linear-diophantine-equations.html}
(Accessed on 25-Oct-2025).

\end{thebibliography}




\end{document}
